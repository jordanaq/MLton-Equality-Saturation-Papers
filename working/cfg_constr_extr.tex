\documentclass{article}
\usepackage{amsmath}
\usepackage{amssymb}
\usepackage{mathpartir}
\usepackage{tikz}
\usepackage{ stmaryrd }
\usepackage{url}

\usetikzlibrary{graphs}

\title{Construction \& Extraction of EqMLton}
\author{Jordan Quinn}
\date{\today}

\begin{document}

\maketitle


\section{Construction}

Given our MLton SSA, specified in ~\cite{MLtonSSA}, a construction for the CFG skeleton, $S$, follows. As our EqSat implementation operates on each function, we look at this structure in our SSA signature. First, we label each block with its index in the function's vector of blocks.

Then, we start at the first block and begin constructing its block in the CFG skeleton. To do this, we construct an e-graph from the statements in this block, mainly identical to the expected construction of a PEG-like e-graph. The most significant difference is that we will create a region for each node that represents the value of an assignment. When a variable already referred to by a region is used to calculate another variable, we create a new region that uses all the parameter nodes for the sub-regions, and we create a new root node that calculates the new variable using the root nodes of the sub-regions as children. This reuses the existing e-graph nodes and limits the parameters to those passed to this block.

Then, when we get to the transfer, we forget all of the regions that were not used.

After the initial construction step, it would be useful to contract sequential blocks until we are left with only merge and join points.

\section{Extraction}

Extraction is more complicated as we need to translate back to MLton SSA at a middle point in its optimization pipeline.


\bibliographystyle{plain}
\bibliography{refs}


\end{document}
